\documentclass[11pt,a4paper]{article}

\usepackage[slovene]{babel}
\usepackage[utf8x]{inputenc}
\usepackage{graphicx}
\usepackage{enumerate}
\usepackage{url}

\pagestyle{plain}

\begin{document}
\title{Poročilo pri predmetu \\
Analiza podatkov s programom R}
\author{Anja Zavrl}
\maketitle

\section{Izbira teme}

Izbrala sem si temo o svetovnih prvakih Formule 1 po letih. Analizirala bom število zmag posameznih ekip in število naslovov dirkačev, na zemljevidu prikazala iz katerih držav prihajajo dirkači in moštva.

Podatki: 
\begin{enumerate}
\item \url{http://en.wikipedia.org/wiki/List_of_Formula_One_World_Drivers%27_Champions#By_season}
\item \url{http://www.formula1.com/results/driver/2014/}
\item \url{http://en.wikipedia.org/wiki/List_of_Formula_One_World_Drivers%27_Champions#By_constructor}
\item \url{http://en.wikipedia.org/wiki/List_of_Formula_One_circuits}
\item \url{http://www.statsf1.com/en/saisons.aspx}
\end{enumerate}

\section{Obdelava, uvoz in čiščenje podatkov}

V tem delu projekta sem obdelala in uvozila podatke o Formuli 1. Uvozila sem dve tabeli iz Excela; to sta tabeli
svetovni prvaki Formule 1 in sezona 2014, in eno tabelo dirketno iz Wikipedije s pomočjo XML; to pa je tabela o konstruktorskih zmagah Formule 1. 

Tabeli svetovni prvaki sem zbrisala nekaj nepomembnih stolpcev in dodala urejenostno spremenljivko, ki pove kolikokrat je bil kakšen dirkač svetovni prvak (enkrat, dvakrat ali večkrat) - to sem uvozila v tabelo Število naslovov dirkača.

Narisala sem tudi dva grafa. Prvi pokaže število točk prvih 5 dirkačev v letošnji sezoni (2014) in je stolpičaste oblike, drugi pa ima obliko pite in pokaže delež konstruktorskih zmag prvih 10 ekip v Formuli 1.

Da zaženemo projekt, moramo zagnati funkcijo \verb|projekt.r|.

\includegraphics[scale = 0.6]{../slike/graf1.pdf}

\includegraphics[scale = 0.5]{../slike/graf2.pdf}

\section{Analiza in vizualizacija podatkov}

V 3. fazi projekta sem uvozila zemljevid sveta, pobarvala države iz katerih prihajajo vsi svetovni prvaki Formule 1 in jih tudi poimenovala. Države, ki imajo več zmag, so pobarvane bolj intenzivno kot tiste, ki imajo manj zmag. 


\makebox[\textwidth][c]{
\includegraphics[width=1.5\textwidth]{../slike/Zmage_po_drzavah.pdf}
}

\section{Napredna analiza podatkov}

Zaradi pomankanja podatkov in neustrezno izbrane teme, sem morala uvoziti še kar nekaj tabel (konstruktorske zmage zadnjih nekaj let). 

\end{document}
