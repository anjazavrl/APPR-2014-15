\documentclass[11pt,a4paper]{article}

\usepackage[slovene]{babel}
\usepackage[utf8x]{inputenc}
\usepackage{graphicx}
\usepackage{enumerate}
\usepackage{url}

\pagestyle{plain}

\begin{document}
\title{Poročilo pri predmetu \\
Analiza podatkov s programom R}
\author{Anja Zavrl}
\maketitle

\section{Izbira teme}

Izbrala sem si temo o svetovnih prvakih Formule 1 po letih. Analizirala bom število zmag posameznih ekip, na zemljevidu prikazala iz katerih držav prihajajo dirkači in moštva in iz katere države prihaja največ dirkačev. 

Podatki: 
\begin{enumerate}
\item \url{http://en.wikipedia.org/wiki/List_of_Formula_One_World_Drivers%27_Champions}
\item \url{http://www.formula1.com/results/driver/2014/}
\end{enumerate}

\section{Obdelava, uvoz in čiščenje podatkov}

V tem delu projekta sem obdelala in uvozila podatke o Formuli 1. Uvozila sem dve tabeli iz Excela; to sta tabeli
svetovni prvaki Formule 1 in sezona 2014, in eno tabelo dirketno iz Wikipedije s pomočjo XML; to pa je tabela o konstruktorskih zmagah Formule 1. 

Tabeli svetovni prvaki sem zbrisala nekaj nepomembnih stolpcev in dodala urejenostno spremenljivko, ki pove kolikokrat je bil kakšen dirkač svetovni prvak (enkrat, dvakrat ali večkrat) - to sem uvozila v tabelo Število naslovov dirkača.

Narisala sem tudi dva grafa. Prvi pokaže število točk prvih 5 dirkačev v letošnji sezoni (2014), drugi pa delež konstruktorskih zmag prvih 10 ekip v Formuli 1.

Da zaženemo projekt, moramo zagnati funkcijo \verb|projekt.r|.

\includegraphics[width=\textwidth]{../slike/graf1.pdf}

\includegraphics[width=\textwidth]{../slike/graf2.pdf}


\section{Analiza in vizualizacija podatkov}

%\includegraphics{../slike/povprecna_druzina.pdf}

\section{Napredna analiza podatkov}

%\includegraphics{../slike/naselja.pdf}

\end{document}
